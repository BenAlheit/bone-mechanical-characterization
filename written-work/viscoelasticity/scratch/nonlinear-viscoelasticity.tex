\documentclass[12pt]{article}
\usepackage[backend=bibtex]{biblatex}

\usepackage{amsmath} % gathered
\usepackage{amssymb} % mathbb
\usepackage{bm} % mathbb
\usepackage{algorithm}% http://ctan.org/pkg/algorithm
\usepackage{algpseudocode}% http://ctan.org/pkg/algorithmicx
\usepackage[margin=0.8in,]{geometry} % Make margins 0.8in inch
\linespread{1.25} % Use 1.5 linespacing (\linespread{x} standard is 1.2, 1.2*x = 1.5 => x = 1.25)
\usepackage{graphicx} % Include images
\usepackage{subcaption} 
\usepackage{color}
\graphicspath{{./figures/}}

\renewcommand{\d}{\text{d}}
\newcommand{\tr}{\text{tr}}
\newcommand{\der}[2]{\dfrac{\text{d} #1}{\text{d} #2}}
\newcommand{\pder}[2]{\dfrac{\partial #1}{\partial #2}}
\newcommand{\sig}{\bm{\sigma}}
\newcommand{\bsig}{\bar{\bm{\sigma}}}
\newcommand{\hsig}{\hat{\bm{\sigma}}}
\newcommand{\eps}{\bm{\varepsilon}}
\newcommand{\beps}{\bar{\bm{\varepsilon}}}
\newcommand{\heps}{\hat{\bm{\varepsilon}}}


\begin{document}

\section{Goals and principals}

Goals:
\begin{itemize}
	\item Develop a numerical framework for implementation of general viscoelasticity constitutive models;
	\item Detail all the required mathematical developments for the model to be defensible against rigorous scrutiny;
	\item Provide a user friendly library which allows a user to implement their own viscoelastic model with a minimum required effort and understanding of the mathematical developments presented here.
\end{itemize}

Principals:
\begin{itemize}
	\item Do as much mathematical development as one can before inserting new information. This creates `modular mathematics' which can be converted into modular code.
	\item If something can be automated it should be automated.
\end{itemize}

\section{A framework for nonlinear generalized 1D rheological Maxwell networks}

\subsection{Kinematics and kinetics}
Consider the network diagram illustrating the viscoelastic behaviour of a 1D constitutive model in Figure \ref{fig:network-diagram}.
\begin{figure}[!htb]
	\centering
%	\includegraphics[width=\columnwidth]{imagefile}
	\caption{ }
	\label{fig:network-diagram}
\end{figure}
The viscoelastic network 

The network diagram implies several axioms for the way in which the components interact:
\begin{enumerate}
	\item The stress of sequential components is equal;
	\item The stress of parallel components is added;
	\item The strain of sequential components is added;
	\item The strain of parallel components is equal.
\end{enumerate} 

Mathematically, these axioms assert that the following equations define the overall constitutive model and that each equation must be satisfied for the constitutive model to be self-consistent:
\begin{align}
	\sigma^{s}_{i} &= \sigma^{d}_{i} = \sigma^{e}_{i}\,,\\
	\sigma & = \sigma^{b} + \sum_{i}^{n}\sigma^{e}_{i}\,,\\
	\varepsilon^{e}_{i} &= \varepsilon^{d}_{i} + \varepsilon^{s}_{i}\,,\label{eq:el-kin-1}\\
	\varepsilon& = \varepsilon^{e}_{i}\,.\label{eq:el-kin-2}
\end{align}

Note that, for the kinematic state of the network to be completely defined, that is, for the strain of each component to be defined $\varepsilon$, $\varepsilon^{d}_{i}$, $\varepsilon^{s}_{i}$, $\varepsilon^{e}_{i}$, $i=1,...,n$, one needs to define $\varepsilon$ and $\varepsilon^{d}_{i}$. Then, the other strain values are defined by equations \eqref{eq:el-kin-1} and \eqref{eq:el-kin-2}. Hence, we identify $\varepsilon^{d}_{i}$ as a state variable. 

\begin{equation}
		\varepsilon = \varepsilon^{d} + \varepsilon^{s}\,.
\end{equation}

\begin{equation}
	\dot{\varepsilon } = \dot{\varepsilon}^{d} + \dot{\varepsilon}^{s}\,.
\end{equation}

\subsection{Thermodynamic considerations}


Rate of work done on each Maxwell branch
\begin{equation}
	\dot{w}^{e}_{i} = \sigma^{e}_{i}\dot{\varepsilon} = \underbrace{\sigma^{s}_{i}\left[\dot{\varepsilon}-\dot{\varepsilon}_{i}^{d}\right]}_{\dot{\Psi}^{e}_{i}} + \underbrace{\sigma^{d}_{i}\dot{\varepsilon}_{i}^{d}}_{\dot{\Phi}^{e}_{i}}
\end{equation}

Axioms for the sub-constitutive models
\begin{enumerate}
	\item $\sigma^{s}_{i}(\varepsilon^{s}_{i})$, $\sigma^{e}_{i}(0) = 0$
	\item $\sigma^{d}_{i}\left(\dot{\varepsilon}^{d}_{i}\right)$,  $\sigma^{d}_{i}\left(0\right) = 0$ 
\end{enumerate}

\begin{equation}
	\der{\Psi^{e}_{i}}{\varepsilon^{s}}\dot{\varepsilon}^{s} = \sigma^{s}_{i}\left[\dot{\varepsilon}-\dot{\varepsilon}_{i}^{d}\right] \Rightarrow \left[\der{\Psi^{e}_{i}}{\varepsilon^{s}} - \sigma^{s}_{i}\right]\dot{\varepsilon}^{s} = 0 \Rightarrow \sigma^{s}_{i} = \der{\Psi^{e}_{i}}{\varepsilon^{s}}
\end{equation}

\begin{equation}
	D = \sigma^{e}_{i}\dot{\varepsilon} - \dot{\Psi}^{e}_{i} \geq 0 
\end{equation}

\begin{equation}
	D = \sigma^{d}_{i}\dot{\varepsilon}_{i}^{d} = \dot{\Phi}^{e}_{i} \geq 0 
\end{equation}

\begin{equation}
	\dot{\Psi} = \dot{\Psi}^{b} + \sum_{i}^{n}  \dot{\Psi}^{e}_{i}
\end{equation}

\subsection{Framework allowing for arbitrary choice of sub-constitutive models}

Partition of the time domain
\begin{equation}
	0 = t_{0} < t_{1} < ... < t_{n} < t_{n+1} = t
\end{equation}
Define
\begin{equation}
	\Delta t_{n+1} = t_{n+1} - t_{n}\,, \qquad t_{n+0.5} = \dfrac{1}{2}\left[t_{n+1} + t_{n}\right]\,.
\end{equation}

\begin{equation}
	\dot{\varepsilon}_{n+1}^{d} = \dfrac{1}{\Delta t_{n+1}}\left[\varepsilon_{n+1}^{d} - \varepsilon_{n}^{d}\right]
\end{equation}

\begin{equation}
	\dot{\varepsilon}_{n+1} = \dfrac{1}{\Delta t_{n+1}}\left[\varepsilon_{n+1} - \varepsilon_{n}\right]
\end{equation}

\begin{equation}
	\sigma^{s}(\varepsilon_{n+1} - \varepsilon_{n+1}^{d}) = \sigma^{d}\left(\dot{\varepsilon}_{n+1}^{d}\right)
\end{equation}

\begin{equation}
		r = \sigma^{d}\left(\dot{\varepsilon}_{n+1}^{d}\right) - \sigma^{s}(\varepsilon_{n+1} - \varepsilon_{n+1}^{d})  
\end{equation}

\begin{equation}
	L[r] = r_{i} + \der{r}{\varepsilon_{n+1(i)}^{d}} \Delta \varepsilon_{n+1(i)}^{d}
\end{equation}

\begin{equation}
	\varepsilon_{n+1(i+1)}^{d} = 	\varepsilon_{n+1(i)}^{d} - \left[\der{r}{\varepsilon_{n+1}^{d}}\right]_{i}^{-1}r_{i}
\end{equation}

\begin{equation}
	\der{r}{\varepsilon_{n+1}^{d}} = \der{\sigma^{d}}{\dot{\varepsilon}^{d}}\der{ }{\varepsilon_{n+1}^{d} }\left[ \dfrac{1}{\Delta t_{n+1}}\left[\varepsilon_{n+1}^{d} - \varepsilon_{n}^{d}\right]\right] - \der{\sigma^{s}}{\varepsilon^{s}}\der{ }{\varepsilon_{n+1}^{d}}\left[\varepsilon_{n+1} - \varepsilon_{n+1}^{d}\right]
\end{equation}


\begin{equation}
	\der{r}{\varepsilon_{n+1}^{d}} = \dfrac{1}{\Delta t_{n+1}}\der{\sigma^{d}}{\dot{\varepsilon}^{d}} + \der{\sigma^{s}}{\varepsilon^{s}}
\end{equation}
-------------------------------------
\begin{equation}
	F = \eta_{1}\dot{d}_{1} = k_{2}d_{2}
\end{equation}

\begin{equation}
	d = d_{1} + d_{2}
\end{equation}

\begin{equation}
\dot{d} = \dot{d}_{1} + \dot{d}_{2}
\end{equation}

\begin{equation}
	\dot{F} = K\dot{d}
\end{equation}

\begin{equation}
k_{2}\dot{d}_{2} = K\left[ \dot{d}_{1} + \dot{d}_{2}\right]
\end{equation}

\begin{equation}
	\eta_{1}\dot{d}_{1} = Kd
\end{equation}

\begin{equation}
	\dot{\sigma}^{e} = \mathbb{C}^{e}\dot{\varepsilon} = \dot{\sigma}^{d} = \dot{\sigma}^{s} = \mathbb{C}^{s} \dot{\varepsilon}^{s} = \mathbb{C}^{s} \left[\dot{\varepsilon} - \dot{\varepsilon}^{d}\right] =  \mathbb{C}^{s} \left[\dot{\varepsilon} - \mathbb{C}^{d-1}\sigma^{e} \right]
\end{equation}
----------------------------

\begin{equation}
\der{\sigma^{e}}{\varepsilon_{n+1}} = \der{ }{\varepsilon_{n+1}}\sigma^{s}(\varepsilon_{n+1} - \varepsilon_{n+1}^{d})  = \der{ }{\varepsilon_{n+1}}\sigma^{d}\left( \dfrac{1}{\Delta t_{n+1}}\left[\varepsilon_{n+1}^{d} - \varepsilon_{n}^{d}\right]\right)
\end{equation}


\begin{equation}
\der{\sigma^{e}}{\varepsilon_{n+1}} = \der{\sigma^{s}}{\varepsilon^{s}_{n+1}}\left[1 - \der{\varepsilon_{n+1}^{d}}{\varepsilon_{n+1}} \right]    = \der{\sigma^{d}}{\dot{\varepsilon}^{d}_{n+1}}\left[ \dfrac{1}{\Delta t_{n+1}}\der{\varepsilon_{n+1}^{d}}{\varepsilon_{n+1}}\right]
\end{equation}

\begin{equation}
\der{\sigma^{s}}{\varepsilon^{s}_{n+1}}       = \left[\der{\sigma^{s}}{\varepsilon^{s}_{n+1}}    +\der{\sigma^{d}}{\dot{\varepsilon}^{d}_{n+1}}\dfrac{1}{\Delta t_{n+1}}\right]\der{\varepsilon_{n+1}^{d}}{\varepsilon_{n+1}}
\end{equation}


\begin{equation}
\der{\varepsilon_{n+1}^{d}}{\varepsilon_{n+1}} = \left[\der{\sigma^{s}}{\varepsilon^{s}_{n+1}}    +\der{\sigma^{d}}{\dot{\varepsilon}^{d}_{n+1}}\dfrac{1}{\Delta t_{n+1}}\right]^{-1}\der{\sigma^{s}}{\varepsilon^{s}_{n+1}} 
\end{equation}

\subsection{Some temporal approximations}

\subsubsection{Backward difference}

\subsubsection{An approximation with chosen order of accuracy}

\begin{equation}
	a\left(t\right) = \sum_{i=0}^{m}\phi_{i}\left(t\right)a_{n+1-i}\,, \qquad \phi_{i} = \prod_{\substack{j=0\\ j\neq i}}^{m}\dfrac{t - t_{n+1-j}}{t_{n+1-i} - t_{n+1-j}}\,.
\end{equation}

\begin{equation}
	\dot{a}\left(t\right) = \sum_{i=0}^{m}\dot{\phi}_{i}\left(t\right)a_{n+1-i}\,, \qquad \dot{\phi}_{i} = \left[\prod_{\substack{j=0\\ j\neq i}}^{m}\dfrac{1}{t_{n+1-i} - t_{n+1-j}}\right]\left[\sum_{\substack{k=0\\ k\neq i}}^{m}\prod_{\substack{j=0\\ j\neq k\\j\neq i}}^{m} t - t_{n+1-j}\right]\,.
\end{equation}

\subsubsection{Stability}

\begin{equation}
	\dot{\sigma}^{d} = \dot{\sigma}^{s}
\end{equation}

\begin{equation}
	\dot{\sigma}^{d} = \der{\sigma^{s}}{\varepsilon^{s}}\dot{\varepsilon}^{s}
\end{equation}

\subsection{Some constitutive models}

\subsubsection{Linear models}

\begin{equation}
	\sigma^{\infty} = E^{\infty}\varepsilon
\end{equation}

\begin{equation}
	\sigma^{s} = E^{s}\left[\varepsilon-\varepsilon^{d}\right]\,, \qquad \sigma^{d} = \eta \dot{\varepsilon}^{d}\,.
\end{equation}

$\sigma^{s} = \sigma^{d} \Rightarrow$

\begin{equation}
	E^{s}\left[\varepsilon-\varepsilon^{d}\right] = \eta \dot{\varepsilon}^{d}
\end{equation}

\begin{equation}
	\tau = \dfrac{\eta}{E^{s}}
\end{equation}

\begin{equation}
	\dfrac{1}{\tau}\varepsilon =  \dfrac{1}{\tau}\varepsilon^{d} + \dot{\varepsilon}^{d}
\end{equation}

Integrating factor:
\begin{equation}
	\exp\left(\int\dfrac{1}{\tau}\,\d t\right) = 	\exp\left(\dfrac{t}{\tau}\right) 
\end{equation}

\begin{equation}
	\dfrac{1}{\tau}\exp\left(\dfrac{t}{\tau}\right) \varepsilon =  \der{ }{t}\left[\exp\left(\dfrac{t}{\tau}\right) \varepsilon^{d} \right] 
\end{equation}

\begin{equation}
	\varepsilon^{d} = \int_{-\infty}^{t}\dfrac{1}{\tau}\exp\left(\dfrac{s-t}{\tau}\right)\varepsilon \,\d s
\end{equation}

IBP

\begin{equation}
		\varepsilon^{d} = \varepsilon - \int_{-\infty}^{t} \exp\left(\dfrac{s-t}{\tau}\right)\dot{\varepsilon} \,\d s
\end{equation}

\begin{equation}
	\sigma^{e} = E^{s}\left[\varepsilon-\varepsilon^{d}\right] =  \int_{-\infty}^{t} \exp\left(\dfrac{s-t}{\tau}\right)E^{s}\dot{\varepsilon} \,\d s
\end{equation}

\begin{equation}
	\sigma = \sigma^{\infty} + \sigma^{e} = \int_{-\infty}^{t} \left[E^{\infty} + \exp\left(\dfrac{s-t}{\tau}\right)E^{s}\right]\dot{\varepsilon} \,\d s
\end{equation}

Integrate by parts ...

Consider three loading conditions
\begin{enumerate}
	\item Instantaneously applied strain:
	\begin{equation}
		\varepsilon = \begin{cases}
			0& t<0\\
			\varepsilon_{0} & 0 \leq t
		\end{cases}
	\end{equation}
	\item Instantaneously applied stress:
	\begin{equation}
		\sigma = \begin{cases}
			0& t<0\\
			\sigma_{0} & 0 \leq t
		\end{cases}
	\end{equation}
	\item Constant strain rate:\begin{equation}
		\varepsilon = \begin{cases}
			0& t<0\\
			\dot{\varepsilon}_{0}t & 0 \leq t
		\end{cases}
	\end{equation}	

	\item Cyclical loading at different rates ...
	
	\item Cyclical loading at different rates incrementally increasing max strain -> damage \& plasticity ...
	
\end{enumerate}

\subsubsection*{Instant strain}
\subsubsection*{Instant stress}
\subsubsection*{Constant strain rate}

Fix incorrect sign:
\begin{equation}
	\sigma = E^{\infty}\dot{\varepsilon}_{0}t - \tau \left[1-\exp\left(\dfrac{-t}{\tau}\right)\right]E^{s}\dot{\varepsilon}_{0} 
\end{equation}


\begin{equation}
	\sigma = E^{\infty}\varepsilon- \tau \left[1-\exp\left(\dfrac{-\varepsilon}{\dot{\varepsilon}_{0} \tau}\right)\right]E^{s}\dot{\varepsilon}_{0} 
\end{equation}

\begin{figure}[!htb]
	\centering
	\begin{subfigure}{0.48\textwidth}
		\centering
		\includegraphics[width=\linewidth]{e-stress}
		\caption{ }
	\end{subfigure}
	\begin{subfigure}{0.48\textwidth}
		\centering
		\includegraphics[width=\linewidth]{e-dot-stress}
		\caption{ }
	\end{subfigure}

	\begin{subfigure}{0.48\textwidth}
		\centering
		\includegraphics[width=\linewidth]{e-dot-stress-tau}
		\caption{ }
	\end{subfigure}
	\begin{subfigure}{0.48\textwidth}
		\centering
		\includegraphics[width=\linewidth]{e-dot-stress-E}
		\caption{ }
	\end{subfigure}
	\caption{Linear viscoelasticity}
\end{figure}

\subsubsection{Ostwald-de Waele model}

Bone is assumed to be impregnated with blood and other biological fluids which are susceptible to shear thinning or thickening

\begin{equation}
\sigma^{d} = \eta(\dot{\epsilon}^{d})\dot{\epsilon}^{d}
\end{equation}

\begin{equation}
D = \eta(\dot{\epsilon}^{d})\left[\dot{\epsilon}^{d}\right]^{2} \geq 0 \,, \qquad \eta(\dot{\epsilon}^{d})\geq 0\,.
\end{equation}

\begin{equation}
\eta(\dot{\epsilon}^{d}) = \eta_{0} |\dot{\epsilon}^{d}|^{n-1}
\end{equation}
Can change slant. See for example Figure \ref{fig:odw}.
\begin{figure}[!htb]
	\centering
	\includegraphics[width=0.65\linewidth]{odw}
	\caption{Ostwald-de Waele viscosity}
	\label{fig:odw}	
\end{figure}

\subsubsection{Modified Ostwald-de Waele model}

ODW model has a numerical issue when $\dot{\epsilon}^{d}=0$. Fix with modification:

\begin{equation}
\eta(\dot{\epsilon}^{d}) = \eta_{0}\left[1+|\dot{\epsilon}^{d}|\right]^{n-1}
\end{equation}
Can change slant. See for example Figure \ref{fig:modw}. Fixes divide by 0 error.
\begin{figure}[!htb]
	\centering
	\includegraphics[width=0.65\linewidth]{modw}
	\caption{Modified Ostwald-de Waele viscosity}
	\label{fig:modw}	
\end{figure}

\subsubsection{Carreau--Yasuda model}

\begin{equation}
	\eta = \eta^{\infty} + \left[\eta^{0}-\eta^{\infty}\right]\left[1 + \left[\lambda \left|\dot{\epsilon}^{d}\right|\right]^{a}\right]^{\left[n-1\right]/a}
\end{equation}


\begin{figure}[!htb]
	\centering
	\begin{subfigure}{0.48\textwidth}
		\centering
		\includegraphics[width=\linewidth]{carreau-lams}
		\caption{ }
	\end{subfigure}
	\begin{subfigure}{0.48\textwidth}
		\centering
		\includegraphics[width=\linewidth]{carreau-etas}
		\caption{ }
	\end{subfigure}
	
	\begin{subfigure}{0.48\textwidth}
		\centering
		\includegraphics[width=\linewidth]{carreau-ns}
		\caption{ }
	\end{subfigure}
	\begin{subfigure}{0.48\textwidth}
		\centering
		\includegraphics[width=\linewidth]{carreau-as} 
		\caption{ }
	\end{subfigure}
	\caption{Carreau--Yasuda viscosity}
\end{figure}

\section{Going 3D}

\subsection{Isochoric-volumetric split of the strain}
Logic: viscous effects are caused by fluids which are approximated as incompressible (only experience isochoric strains).


\begin{equation}
\bm{\sigma}^{\infty} = \lambda \bm{I}\tr\left(\bm{\varepsilon}\right) + 2\mu \bm{\varepsilon}
\end{equation}

\begin{equation}
\sigma_{ij}^{\infty} = \lambda \delta_{ij}\varepsilon_{mm} + 2\mu \varepsilon_{ij}
\end{equation}


\begin{equation}
\bm{\varepsilon} = \bar{\bm{\varepsilon}} + \hat{\bm{\varepsilon}} \,, \qquad \hat{\bm{\varepsilon}} = \dfrac{1}{3}\bm{I} \tr\left(\bm{\varepsilon}\right)
\end{equation}

\begin{equation}
\bm{\sigma}^{\infty} = \lambda \bm{I}\tr\left(\bar{\bm{\varepsilon}} + \hat{\bm{\varepsilon}}\right) + 2\mu \left[\bar{\bm{\varepsilon}} + \hat{\bm{\varepsilon}}\right]
\end{equation}

\begin{equation}
\sigma_{ij}^{\infty} = \lambda \delta_{ij} \hat{\varepsilon}_{mm} + 2\mu \hat{\varepsilon}_{ij} + 2\mu \bar{\varepsilon}_{ij}
\end{equation}


\begin{equation}
\begin{aligned}
\sigma_{ij}^{\infty} = \underbrace{\left[\lambda \delta_{ij} \delta_{kl}  + 2\mu \delta_{ik} \delta_{jl}\right]}_{\hat{\mathbb{C}}_{ijkl}} \hat{\varepsilon}_{kl} + \underbrace{2\mu \delta_{ik} \delta_{jl}}_{\bar{\mathbb{C}}_{ijkl}} \bar{\varepsilon}_{kl}\\
\hat{\sigma}_{ij}^{\infty} = \hat{\mathbb{C}}_{ijkl}\hat{\varepsilon}_{kl}\,,\qquad \bar{\sigma}_{ij}^{\infty}= \bar{\mathbb{C}}_{ijkl} \bar{\varepsilon}_{kl}
\end{aligned}
\end{equation}


\subsection{Maxwell network in 3D}
Motivated by 1D Maxwell network:
\begin{align}
	\bar{\bm{\sigma}}^{s}_{i} &= \bar{\bm{\sigma}}^{d}_{i} = \bar{\bm{\sigma}}^{e}_{i}\,,\\
	\bar{\bm{\varepsilon}}_{i} &= \bar{\bm{\varepsilon}}^{d}_{i} + \bar{\bm{\varepsilon}}^{s}_{i}\,,\\
	\bar{\bm{\sigma}} & = \bar{\bm{\sigma}}^{b} + \sum_{i}^{n}\bar{\bm{\sigma}}^{e}_{i}\,,\\
	\hat{\bm{\sigma}}^{s}_{i} &= \hat{\bm{\sigma}}^{d}_{i} = \hat{\bm{\sigma}}^{e}_{i}\,,\\
	\hat{\bm{\varepsilon}}_{i} &= \hat{\bm{\varepsilon}}^{d}_{i} + \hat{\bm{\varepsilon}}^{s}_{i}\,,\\
	\hat{\bm{\sigma}} & = \hat{\bm{\sigma}}^{b} + \sum_{i}^{n}\hat{\bm{\sigma}}^{e}_{i}\,,\\
	\bm{\sigma} &=\hat{\bm{\sigma}} +\bar{\bm{\sigma}}\,.
\end{align}

\subsubsection{Thermodynamic considerations}
{\color{red} Messy and tricky... something to consider... actually this is prior to the choice of subconstitutive models. Probably been addressed before. Go look in the literature.}

\begin{equation}
	D = \bm{\sigma}:\bm{\varepsilon} - \dot{\Psi} \geq 0
\end{equation}

\begin{equation}
	D = \left[\bm{\sigma}^{b} + \sum_{i}^{n}\bm{\sigma}^{e}_{i}\right] :\bm{\varepsilon} - \dot{\Psi} \geq 0
\end{equation}

\begin{equation}
	\left[\bsig^{e} + \hsig^{e}\right] : \left[\beps^{e} + \heps^{e}\right] = \bsig^{e}:\beps^{e} + \hsig^{e}:\beps^{e} + \bsig^{e} : \heps^{e} + \hsig^{e}: \heps^{e}
\end{equation}

\begin{equation}
	\bsig^{e}:\eps = \bsig^{e}:\left[\beps + \heps\right] = \bsig^{e}:\beps + \bsig^{e}:\heps = \bsig^{s}:\beps^{s} + \bsig^{d}:\beps^{d} + \bsig^{e}:\heps
\end{equation}

\begin{equation}
D_{i} = \bar{\bm{\sigma}}_{i}^{s}:\bar{\bm{\varepsilon}}_{i}^{s} - \dot{\Psi}
\end{equation}


\subsubsection{Framework}

%
%\begin{algorithm}[!htb]
%	\caption{Procedure for determining stress and constitutive tensor required by {\sc abaqus} for a compressible material given the derivatives of its strain energy function.}
%	\begin{algorithmic}[1]
%		\Procedure{increment\_viscoelastic\_material}{$\bm{\varepsilon}^{n+1}$, $\bar{\bm{\varepsilon}}^{d(n)}$}
%		\State $\hat{\bm{\varepsilon}} = \dfrac{1}{3}\bm{I}\tr\left(\bm{\varepsilon}^{n+1}\right) $
%		\State $\bar{\bm{\varepsilon}} = \bm{\varepsilon}^{n+1} - \hat{\bm{\varepsilon}}$
%		\State $\bm{\sigma}^{\infty} = 3\lambda\hat{\bm{\varepsilon}} + 2 \mu \bm{\varepsilon}^{n+1}$
%		\State $\bar{\bm{\sigma}}^{\infty}$(:, :, :) = \Call{CALC\_BIG\_A}{$\bm{F}$, $\bm{a}_{0}$, $\bm{b}_{0}$}
%		\State $J = \det \bm{F}$
%		\State $\bm{\sigma} = \bm{0}$
%		
%		\State \Return $\bm{\sigma}$, $\mathbb{c}_{\text{\sc aba}}$
%		\EndProcedure
%	\end{algorithmic}
%\end{algorithm}


\subsubsection{Tangent}

\begin{equation}
	\pder{\bar{\bm{\sigma}}^{e}}{\bm{\varepsilon}_{n+1}} = \pder{}{\bm{\varepsilon}_{n+1}}\bar{\bm{\sigma}}^{s}\left(\bar{\bm{\varepsilon}}_{n+1} - \bar{\bm{\varepsilon}}^{d}_{n+1}\right) = \pder{}{\bm{\varepsilon}_{n+1}}\bar{\bm{\sigma}}^{d}\left(\dfrac{1}{\Delta t}\left[\bar{\bm{\varepsilon}}^{d}_{n+1} - \bar{\bm{\varepsilon}}^{d}_{n}\right]\right) 
\end{equation}

\begin{equation}
\pder{\bar{\bm{\sigma}}^{e}}{\bm{\varepsilon}_{n+1}} = \pder{\bar{\bm{\sigma}}^{s}}{\bar{\bm{\varepsilon}}^{s}} \left[\pder{\bar{\bm{\varepsilon}}_{n+1}}{\bm{\varepsilon}_{n+1}} - \pder{\bar{\bm{\varepsilon}}^{d}_{n+1}}{\bm{\varepsilon}_{n+1}}\right] = \pder{\bar{\bm{\sigma}}^{d}}{\dot{\bar{\bm{\varepsilon}}}^{d}} \left[\dfrac{1}{\Delta t}\pder{\bar{\bm{\varepsilon}}^{d}_{n+1}}{\bm{\varepsilon}_{n+1}} \right]
\end{equation}


\begin{equation}
\pder{\bar{\bm{\sigma}}^{e}}{\bm{\varepsilon}_{n+1}} = \bar{\mathbb{C}}^{s} \left[\pder{\bar{\bm{\varepsilon}}_{n+1}}{\bm{\varepsilon}_{n+1}} - \pder{\bar{\bm{\varepsilon}}^{d}_{n+1}}{\bm{\varepsilon}_{n+1}}\right] =  \pder{\bar{\bm{\sigma}}^{d}}{\dot{\bar{\bm{\varepsilon}}}^{d}}\left[\dfrac{1}{\Delta t}\pder{\bar{\bm{\varepsilon}}^{d}_{n+1}}{\bm{\varepsilon}_{n+1}} \right]
\end{equation}

\begin{equation}
\bar{\mathbb{C}}^{s} \pder{\bar{\bm{\varepsilon}}_{n+1}}{\bm{\varepsilon}_{n+1}}  = \left[\bar{\mathbb{C}}^{s} +  \pder{\bar{\bm{\sigma}}^{d}}{\dot{\bar{\bm{\varepsilon}}}^{d}}\dfrac{1}{\Delta t}\right]\pder{\bar{\bm{\varepsilon}}^{d}_{n+1}}{\bm{\varepsilon}_{n+1}} 
\end{equation}

\begin{equation}
\pder{\bar{\bm{\varepsilon}}^{d}_{n+1}}{\bm{\varepsilon}_{n+1}}  = \left[\bar{\mathbb{C}}^{s} +  \pder{\bar{\bm{\sigma}}^{d}}{\dot{\bar{\bm{\varepsilon}}}^{d}}\dfrac{1}{\Delta t}\right]^{-1}\bar{\mathbb{C}}^{s} \pder{\bar{\bm{\varepsilon}}_{n+1}}{\bm{\varepsilon}_{n+1}}  
\end{equation}


\begin{equation}
\pder{\bar{\bm{\sigma}}^{e}}{\bm{\varepsilon}_{n+1}}  = \dfrac{1}{\Delta t} \pder{\bar{\bm{\sigma}}^{d}}{\dot{\bar{\bm{\varepsilon}}}^{d}}\left[\bar{\mathbb{C}}^{s} +  \pder{\bar{\bm{\sigma}}^{d}}{\dot{\bar{\bm{\varepsilon}}}^{d}}\dfrac{1}{\Delta t}\right]^{-1}\bar{\mathbb{C}}^{s} \pder{\bar{\bm{\varepsilon}}_{n+1}}{\bm{\varepsilon}_{n+1}}  
\end{equation}

\begin{equation}
	\pder{\bar{\bm{\varepsilon}}_{n+1}}{\bm{\varepsilon}_{n+1}}  = \bm{I} \odot \bm{I} - \dfrac{1}{3}\bm{I} \otimes \bm{I} \,.
\end{equation}

Can verify later with differencing procedure

\subsubsection{Algorithm for implementation}

\subsection{Some models}

\subsubsection{Linear models}
\begin{equation}
	\bar{\sigma}_{ij}^{\infty}= \bar{\mathbb{C}}_{ijkl} \bar{\varepsilon}_{kl}
\end{equation}

\begin{equation}
	\bar{\sigma}_{ij}^{d} = \bar{\mathbb{C}}^{d}_{ijkl} \dot{\bar{\varepsilon}}^{d}_{kl}
\end{equation}

\begin{equation}
\bar{\sigma}_{ij}^{s} = \bar{\mathbb{C}}^{s}_{ijkl} \left[\bar{\varepsilon}_{kl}-\bar{\varepsilon}^{d}_{kl}\right]
\end{equation}

\begin{equation}
	\bar{\mathbb{C}}^{d}_{ijkl} \dot{\bar{\varepsilon}}^{d}_{kl} = \bar{\mathbb{C}}^{s}_{ijkl} \left[\bar{\varepsilon}_{kl}-\bar{\varepsilon}^{d}_{kl}\right]
\end{equation}

\begin{equation}
\bar{\mathbb{C}}^{d}_{ijkl} \dot{\bar{\varepsilon}}^{d}_{kl} + \bar{\mathbb{C}}^{s}_{ijkl}\bar{\varepsilon}^{d}_{kl} = \bar{\mathbb{C}}^{s}_{ijkl} \bar{\varepsilon}_{kl}
\end{equation}

\begin{equation}
	\bar{\mathbb{C}}^{d} = \tau\bar{\mathbb{C}}^{s}
\end{equation}

\begin{equation}
\dot{\bar{\bm{\varepsilon}}}^{d} + \dfrac{1}{\tau}\bar{\bm{\varepsilon}}^{d} = \dfrac{1}{\tau} \bar{\bm{\varepsilon}}
\end{equation}


\begin{equation}
\bar{\bm{\varepsilon}}^{d} = \bar{\bm{\varepsilon}} - \int_{-\infty}^{t} \exp\left(\dfrac{s-t}{\tau}\right)\dot{\bar{\bm{\varepsilon}}} \,\d s
\end{equation}

\begin{equation}
\bar{\bm{\sigma}}^{e} = \int_{-\infty}^{t} \exp\left(\dfrac{s-t}{\tau}\right)\bar{\mathbb{C}}^{s}\dot{\bar{\bm{\varepsilon}}} \,\d s
\end{equation}

\subsubsection{These special visc models}

\begin{equation}
	\bar{\bm{\sigma}}^{d} = \eta\left(\left|\left|\dot{\bar{\bm{\varepsilon}}}^{d}\right| \right|\right)\dot{\bar{\bm{\varepsilon}}}^{d}
\end{equation}

\begin{equation}
	\pder{\bar{\bm{\sigma}}^{d}}{\dot{\bar{\bm{\varepsilon}}}^{d}}= \eta\left(\left|\left|\dot{\bar{\bm{\varepsilon}}}^{d}\right| \right|\right) \bm{I}\odot\bm{I} +\pder{\eta}{\left|\left|\dot{\bar{\bm{\varepsilon}}}^{d}\right| \right|} \dot{\bar{\bm{\varepsilon}}}^{d} \otimes \pder{\left|\left|\dot{\bar{\bm{\varepsilon}}}^{d}\right| \right|}{\dot{\bar{\bm{\varepsilon}}}^{d} }
\end{equation}

\begin{equation}
	\pder{}{\dot{\bar{\varepsilon}}^{d}_{ij}}\left[\dot{\bar{\varepsilon}}^{d}_{mn}\dot{\bar{\varepsilon}}^{d}_{mn}\right]^{1/2} = \left[\dot{\bar{\varepsilon}}^{d}_{mn}\dot{\bar{\varepsilon}}^{d}_{mn}\right]^{-1/2} \dot{\bar{\varepsilon}}^{d}_{ij}
\end{equation}


\begin{equation}
\pder{\bar{\bm{\sigma}}^{d}}{\dot{\bar{\bm{\varepsilon}}}^{d}}= \eta\left(\left|\left|\dot{\bar{\bm{\varepsilon}}}^{d}\right| \right|\right) \bm{I}\odot\bm{I} +\pder{\eta}{\left|\left|\dot{\bar{\bm{\varepsilon}}}^{d}\right| \right|} \left|\left|\dot{\bar{\bm{\varepsilon}}}^{d}\right| \right|^{-1}\dot{\bar{\bm{\varepsilon}}}^{d} \otimes \dot{\bar{\bm{\varepsilon}}}^{d}
\end{equation}

\subsubsection{Ostwald-de Waele model}

\begin{equation}
	\bar{\bm{\sigma}}^{d} = \bar{\mathbb{C}}^{d}\dot{\bar{\bm{\varepsilon}}}^{d}
\end{equation}

\begin{equation}
	 \bar{\mathbb{C}}^{d}_{ijop} = \eta_{0} \left|\left|\dot{\bar{\bm{\varepsilon}}}^{d}\right|\right|^{n-1} \delta_{io}\delta_{jp}
\end{equation}

\begin{equation}
	\pder{\bar{\bm{\sigma}}^{d}_{ij}}{\dot{\bar{\bm{\varepsilon}}}_{kl}^{d}} = \bar{\mathbb{C}}^{d}_{ijkl} + \left[\pder{ }{\dot{\bar{\bm{\varepsilon}}}_{kl}^{d}} \bar{\mathbb{C}}^{d}_{ijop}\right]\dot{\bar{\bm{\varepsilon}}}_{op}^{d}
\end{equation}

\begin{equation}
	\pder{ }{\dot{\varepsilon}_{kl}}\left(\dot{\varepsilon}_{mn}\dot{\varepsilon}_{mn}\right)^{\left[n-1\right]/2} = \left[n-1\right]\left(\dot{\varepsilon}_{mn}\dot{\varepsilon}_{mn}\right)^{\left[n-3\right]/2}\dot{\varepsilon}_{kl} = \left[n-1\right]\left|\left|\dot{\bm{\varepsilon}}\right |\right|^{n-3}\dot{\varepsilon}_{kl}
\end{equation}

\begin{equation}
		\pder{\bar{\bm{\sigma}}^{d}_{ij}}{\dot{\bar{\bm{\varepsilon}}}_{kl}^{d}} = \bar{\mathbb{C}}^{d}_{ijkl} +\left[n-1\right]\left|\left|\dot{\bar{\bm{\varepsilon}}}^{d}\right |\right|^{n-3}\dot{\bar{\varepsilon}}^{d}_{ij}\dot{\bar{\varepsilon}}^{d}_{kl}
\end{equation}


\begin{equation}
\pder{\bar{\bm{\sigma}}^{d}}{\dot{\bar{\bm{\varepsilon}}}^{d}} = \bar{\mathbb{C}}^{d} +\left[n-1\right]\left|\left|\dot{\bar{\bm{\varepsilon}}}^{d}\right |\right|^{n-3}\dot{\bar{\bm{\varepsilon}}}^{d}\otimes\dot{\bar{\bm{\varepsilon}}}^{d}
\end{equation}

\begin{equation}
	\bm{r} = \bar{\bm{\sigma}}^{d}\left(\dot{\bar{\bm{\varepsilon}}}_{n+1}^{d}\right) - \bar{\bm{\sigma}}^{s}(\bar{\bm{\varepsilon}}_{n+1} - \bar{\bm{\varepsilon}}_{n+1}^{d})  
\end{equation}

\begin{equation}
	\pder{\bm{r}}{\bar{\bm{\varepsilon}}_{n+1}^{d}} = \pder{\bar{\bm{\sigma}}^{d}\left(\dot{\bar{\bm{\varepsilon}}}_{n+1}^{d}\right)}{\dot{\bar{\bm{\varepsilon}}}_{n+1}^{d}}\pder{\dot{\bar{\bm{\varepsilon}}}_{n+1}^{d}}{\bar{\bm{\varepsilon}}_{n+1}^{d}} + \pder{\bar{\bm{\sigma}}^{s}}{\bar{\bm{\varepsilon}}_{n+1}^{s}}
\end{equation}


\end{document}
