\documentclass[12pt]{article}


%% Packages
\usepackage{algorithm}% http://ctan.org/pkg/algorithm
\usepackage{algpseudocode}% http://ctan.org/pkg/algorithmicx
\usepackage{amsmath} % gathered
\usepackage{amssymb} % mathbb
\usepackage{bm} % mathbb
\usepackage{color}
\usepackage[backend=bibtex]{biblatex}
\usepackage[margin=0.8in,]{geometry} % Make margins 0.8in inch
\linespread{1.25} % Use 1.5 linespacing (\linespread{x} standard is 1.2, 1.2*x = 1.5 => x = 1.25)
\usepackage{graphicx} % Include images
\usepackage{nomencl}
\makenomenclature
\usepackage{subcaption} 


%% Defs

\renewcommand{\d}{\text{d}}
\newcommand{\tr}{\text{tr}}
\newcommand{\der}[2]{\dfrac{\text{d} #1}{\text{d} #2}}
\newcommand{\pder}[2]{\dfrac{\partial #1}{\partial #2}}
\newcommand{\sig}{\bm{\sigma}}
\newcommand{\bsig}{\bar{\bm{\sigma}}}
\newcommand{\hsig}{\hat{\bm{\sigma}}}
\newcommand{\eps}{\bm{\varepsilon}}
\newcommand{\beps}{\bar{\bm{\varepsilon}}}
\newcommand{\heps}{\hat{\bm{\varepsilon}}}

\graphicspath{{./figures/}}
\bibliography{nonlinear-viscoelasticity.bib}

\begin{document}

\tableofcontents

\begin{abstract}
	{\color{red}
	\begin{itemize}
		\item Linear viscoelastic models are not able to fit full strain rate uniaxial loading data for bone well \cite{Cloete2014}. Furthermore, biomechanics and the characterization of biological materials is rapidly evolving. As biological solids are impregnated with biological fluids which typically display non-Newtonian behaviour it stands to reason that many biological materials which are yet to be mechanically characterized could display non-linear viscoelastic behaviour. Hence, the existence of a flexible framework for non-linear viscoelasticity is an imperative in the field of biomechanics.
		\item A framework of non-linear viscoelasticity in an infinitesimal setting does not appear to exist in the literature and so one is developed here.
		\item A generalized Maxwell model is adopted without \textit{a priori} assumption of the constitutive behaviour of the components in the Maxwell elements.
		\item The crux is to enforce that viscous and elastic stresses in the dashpot and spring, respectively, are equal. In an incremental (temporally discretized) setting this is achieved by constructing a residual function and applying local (integration point level) Newton-Raphson iterations to determine the viscous strain (that is, the strain of the dashpot) at the next iteration such that the viscous and elastic stresses are equal.
		\item This framework can be extended to finite strains and possesses some advantages over current finite strain non-linear viscoelastic models \cite{Reese1998}. In particular, the model need not be formulated in terms of principal stretches. 
	\end{itemize}
}
\end{abstract}	
	
	\mbox{}
	
	\nomenclature[P]{\(c\)}{Speed of light in a vacuum}
	\nomenclature[P]{\(h\)}{Planck constant}
	\nomenclature[P]{\(G\)}{Gravitational constant}
	\nomenclature[N]{\(\mathbb{R}\)}{Real numbers}
	\nomenclature[N]{\(\mathbb{C}\)}{Complex numbers}
	\nomenclature[N]{\(\mathbb{H}\)}{Quaternions}
	\nomenclature[O]{\(V\)}{Constant volume}
	\nomenclature[O]{\(\rho\)}{Friction index}
	
	\printnomenclature

\section{Introduction}	


\section{Small strain non-linear viscoelasticity in 1D}

\subsection{Maxwell model kinetics and kinematics}

\subsection{Thermodynamic considerations}

\subsection{Temporal discretization}

\subsection{The framework}

\subsubsection{Satisfying the axioms of the Maxwell model}

\subsection{Some models and their behaviour}

\subsubsection{Linear viscoelasticity}

\subsubsection{Ostwald-de Waele model}

\subsubsection{Modified Ostwald-de Waele model}

\subsubsection{Carreau--Yasuda model}

\subsubsection{Model behaviour}

\subsubsection*{Instant stress}
\subsubsection*{Instant strain}
\subsubsection*{Constant strain rate}


\subsection{Fitting models to data}

\section{Small strain non-linear viscoelasticity in 3D}

\subsection{Isochoric--volumetric split}

\subsection{Maxwell model kinetics and kinematics}

\subsection{Thermodynamic considerations}

\subsection{The framework \label{sec:frame-work-1d}}

\subsubsection{Satisfying the axioms of the Maxwell model}

\subsection{Some models and their behaviour}

We focus on the viscous component of the model assuming linear behaviour for the elastic component. However, linear behaviour of the elastic component is not a requirement for the framework presented in Section \ref{sec:frame-work-1d}. We take inspiration from non-Newtonian models commonly used for biological fluids.

\subsubsection{Linear viscoelasticity}

\subsubsection{Ostwald-de Waele model}

\subsubsection{Modified Ostwald-de Waele model}

\subsubsection{Carreau--Yasuda model}

\subsubsection{Model behaviour}

\subsubsection*{Instant stress}
\subsubsection*{Instant strain}
\subsubsection*{Constant strain rate}

\subsection{Fitting models to data}

\section{Computational examples}

\subsection{Inverse FE to fit models??}

\subsection{Impact loading on bone: comparison of results for different models}


\printbibliography

\appendix

\section{Other temporal approximations}

\section{Why it's incorrect to simply insert a new viscosity at each time-step}

\section{Some more detailed maths (maybe)}

\section{Extension to finite strain}

\end{document}